\documentclass{article}

\setlength{\parskip}{\baselineskip}
\setlength{\parindent}{0pt}
\usepackage[fontsize=12pt]{fontsize}
\usepackage{graphicx}
\usepackage[left=2cm,right=2cm,top=2cm,bottom=2cm]{geometry}
\usepackage{fancyhdr}
\usepackage{xcolor}
\usepackage{lastpage}
\usepackage{caption}
\usepackage{amsmath}
\usepackage{bm}
\pagestyle{fancy}
\renewcommand{\headrulewidth}{0pt}
\fancyhead{}
\cfoot{Page \thepage\ of \pageref{LastPage}}
\thispagestyle{fancy}
\setlength{\parindent}{0pt}


\begin{document}


\section*{\begin{center}
    Documentation of the Optional Square Wave Transportation Assessment
\end{center}}
\stepcounter{section}


\subsection{Summary}
The file ``wave\rule{2mm}{0.15mm}transport.c'' contains the C code
for transporting a square wave across a 1-D structured grid. The file 
``create\rule{2mm}{0.15mm}gif.gp'' uses the output of the C code to create a 
10 second GIF of the square wave being transported.

\textbf{Note:} Access to the research paper by Boris \& Book referenced 
in the problem statement was restricted behind paywalls. Nevertheless, an 
attempt was made to transport the square wave as described. The flux-
corrected method was not successfully implemented based on initial research 
found online, but a method with diffusion was successfully implemented. 

Example usage in Windows command prompt:
\vspace{-5mm}
\begin{itemize}
    \item Build: ``make''
    \item Run: ``wave\rule{2mm}{0.15mm}transport.exe \&\& 
    gnuplot create\rule{2mm}{0.15mm}gif.gp''
\end{itemize}

Example output:
\vspace{-5mm}
\begin{center}
    \fbox{ \includegraphics[width=0.75\textwidth]{Capture.PNG} }
\end{center}

\newpage
\stepcounter{section}
\subsection{Implementation Methodology}
The code implemented in the C file is a simple methodology using the Leapfrog 
technique to update the continuity and momentum equations. The following 
steps are taken:
\begin{enumerate}
    \item Initialize the square wave to a density of 25.0 in cells 50 to 149.
    \item Loop over the desired time span.
    \begin{enumerate}
        \item Update the flux and density via continuity equation over a half time step.
        \item Use the density from 2(a) to update the momentum over a full time step.
        \item Use the updated momentum from 2(b) to update the density over a full time step.
        \item Save density data to a frame to be used in the final GIF.
    \end{enumerate}
    \item Combine all frames into one GIF.
\end{enumerate}



\end{document}